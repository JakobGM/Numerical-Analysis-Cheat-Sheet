\section{DFT/FFT}

You should have a pretty good idea about the definition of DFT and be able to apply it, and derive its properties. Of course different definitions use different normalization constants, and the definition used in the book is absolutely symmetric with respect to direct/inverse transforms (I can never remember which one is which), so simply state the definitions of direct/inverse transforms that you are using at the exam! Idea about FFT and its complexity. Connection between Fourier transform, trigonometric interpolation, and a general idea about filtering/compression (~ at the level of the project).