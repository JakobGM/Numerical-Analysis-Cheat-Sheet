\section{DFT/FFT}
%You should have a pretty good idea about the definition of DFT and be able to apply it, and derive its properties. Of course different definitions use different normalization constants, and the definition used in the book is absolutely symmetric with respect to direct/inverse transforms (I can never remember which one is which), so simply state the definitions of direct/inverse transforms that you are using at the exam! 
\begin{definition}
The \textbf{Discrete Fourier Transform} of $x = [x_0,...,x_{n-1}]^T$ is the $n$-dimensional vector $y = [y_0,...,y_{n-1}]^T$, where $w = e^{-i2\pi/n}$ and
$$
y_k = \frac{1}{\sqrt{n}} \sum_{j=0}^{n-1}x_j w^{jk}.
$$
\end{definition}

Or in matrix terms
\begin{gather*}
\begin{bmatrix}
    y_0 \\
    y_1 \\
    y_2 \\
    \vdots \\
    y_{n-1}
\end{bmatrix}
=
\begin{bmatrix}
    a_0 + ib_0 \\
    a_0 + ib_1 \\
    a_0 + ib_2 \\
    \vdots \\
    a_{n-1} + ib_{n-1}
\end{bmatrix}
=
F_n
\begin{bmatrix}
    x_0 \\
    x_1 \\
    x_2 \\
    \vdots \\
    x_{n-1}
\end{bmatrix},
\end{gather*}

where the \textbf{Fourier matrix}, $F_n$, is equal to
$$
F_n = \frac{1}{\sqrt{n}}
\begin{bmatrix}
    w^0 & w^0 & w^0 & \hdots & w^0 \\
    w^0 & w^1 & w^2 & \hdots & w^{n-1} \\
    w^0 & w^2 & w^4 & \hdots & w^{2(n-1)} \\
    w^0 & w^3 & w^6 & \hdots & w^{3(n-1)} \\
    \vdots & \vdots & \vdots & \ddots & \vdots \\
    w^0 & w^{n-1} & w^{2(n-1)} & \hdots & w^{(n-1)^2}
\end{bmatrix}
$$
The \textbf{inverse Discrete Fourier Transform} is then given by $x = F_n^{-1}y$ or 
$$
x_k = \frac{1}{\sqrt{n}} \sum_{j=0}^{n-1}y_j w^{-jk}.
$$

\subsection{Useful properties of the DFT}
\begin{itemize}
    \item The inverse of the Fourier matrix is the matrix consisting of the complex conjugates of the entries of $F_n$: $F_n^{-1} = \conj{F}_n$.
    \item The Fourier matrix is \textbf{unitary}, that is $\conj{F}_n^T F_n = I$.
    \item The magnitude of a complex vector is: $\magn{\vec{v}} = \sqrt{\conj{\vec{v}}^T \vec{v}}$.
    \item There is no change in magnitude after a unitary matrix multiplication: $\magn{Fv}^2 = \conj{v}^T\conj{F}^T F v = \conj{v}^T v = \magn{v}^2$.
    \item If $\vec{x}$ is real, then $y_0$ is real, and $y_{n-k} = \conj{y_k}$.
    \item $\vec{x_1} \cdot \vec{x_2} = \vec{x_2}^T \vec{x_1}$ and $[F_n \vec{x}]^T = \vec{x}^TF_n^{-1}$.
\end{itemize}

\subsection{The Fast Fourier Transform}
%Idea about FFT and its complexity. 
The FFT uses the following property in order to split DFT$(N)$ into two DFT$(N/2)$s plus $2N-1$ extra operations

\begin{gather*}
 \sum_{n=0}^{N-1}x_n e^{-2 \pi i n k / N} \\
= \sum_{n=0}^{N/2-1} x_{2n}e^{-2 \pi i(2n)k/N} + \sum_{n=0}^{N/2-1} x_{2n+1}e^{-2 \pi i(2n+1)k/N} \\
= \sum_{n=0}^{N/2-1} x_n^{\text{even}}e^{-2 \pi i n k /(N/2)} + e^{-2 \pi i k / N} \sum_{n=0}^{N/2-1} x_n^{\text{odd}}e^{-2 \pi i n k /(N/2)}
\end{gather*}

\subsection{Trigonometric interpolation}
%Connection between Fourier transform, trigonometric interpolation, ...
Given the interval $[c,d]$ and positive integer $n$, let $t_j = c + j(d-c)/n$ for $j = 0,...,n-1$, and let $\vect{x}=(x_0,...,x_{n-1})$ denote a vector of $n$ numbers. Define $\vec{a} + \vec{b}i = F_n \vec{x}$. Then the complex function
$$
Q(t) = \frac{1}{\sqrt{n}} \sum_{k=0}^{n-1} (a_k + i b_k) e^{i 2\pi k (t-c)/(d-c)}
$$
satisfies $Q(t_j) = x_j$ for $j= 0,...,n-1$. Furthermore, if the $x_j$ are real, the function
\begin{gather*}
P(t) = \frac{1}{\sqrt{n}} \sum_{k=0}^{n-1} \left( a_k \cos{\frac{2 \pi k (t-c)}{d-c}} - b_k \sin{\frac{2 \pi k (t-c)}{d-c}} \right)
\end{gather*}
satisfies $P(t_j) = x_j$ for $j = 0, ... ,n-1$, assuming $n$ is even. Using the cosine and sine addition formulas together with the fact that $y_{n-k} = \conj{y_k}$, $P(t)$ can be simplified to

\begin{gather*}
    P_n(t) = \frac{a_0}{\sqrt{n}}  
    + \frac{2}{\sqrt{n}} \sum_{k=1}^{n/2-1} \left( a_k \cos{\frac{2 \pi k (t-c)}{d-c}} - b_k \sin{\frac{2 \pi k (t-c)}{d-c}} \right) \\
    + \frac{a_{n/2}}{\sqrt{n}} \cos{\frac{n \pi (t-c)}{d-c}}
\end{gather*}

\subsection{Fourier filtering/compression relevant to the project}
%... and a general idea about filtering/compression (~ at the level of the project).
\begin{itemize}
    \item \mcode{MATLAB} uses a non-unitary normalization for its Fourier transformation, such that $F_n x$ is computed by \mcode{fft(x)/sqrt(n)}, and $F_n^{-1} y$ by \mcode{ifft(y)*sqrt(n)}.
    \item Given $n$ data points, the best least squares trigonometric function with $m < n$ terms can be found by interpolating with $n$ terms, and then only keep the first $m$ terms (dropping the higher frequencies, called a \textbf{low pass filter}).
    \item A \textbf{high pass filter} can be made by dropping the lower frequency components.
\end{itemize}